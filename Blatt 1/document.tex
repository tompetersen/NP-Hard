\documentclass[12pt,a4paper]{article}
\usepackage[utf8]{inputenc}
\usepackage{amsmath}
\usepackage{amsfonts}
\usepackage{amssymb}
\usepackage{graphicx}
\usepackage{color}
\usepackage{listings} 
 \definecolor{lightgrey}{rgb}{0.90,0.90,0.90}
\lstset{language=Java, backgroundcolor=\color{lightgrey},  numbersep=5pt, tabsize=3}
 \usepackage{lineno}
\title{Lösungsstrategien für NP-schwere Probleme\\Aufgabenblatt 01}

\author{
		Jakob Rieck\\
		\small{6423721}
	\and
		Konstantin Kobs\\
		\small{6414943}
	\and
		Thomas Maier\\
		\small{6319878}
	\and
		Tom Petersen\\
		\small{6359640}
}
\date{Abgabe zum 18.04.16}

\begin{document}

\maketitle

\subsection*{Aufgabe 1}

\subsubsection*{a)}
Zunächst formulieren wir das 4D-Matching-Problem:\\
\textbf{Eingabe:} Disjunkte Mengen $A$, $B$, $C$ und $D$ mit $|A| = |B| = |C| = |D| = n$ sowie eine Menge $T \subseteq A \times B \times C \times D$ von Tripeln.\\
\textbf{Frage:} Gibt es eine Menge von $n$ Tripeln in $T$, sodass jedes Element aus $A \cup B \cup C \cup D$ in genau einem dieser Tripel vorkommt?
\\\\
Nun zeigen wir, dass 4D-Matching in NP liegt. Dazu muss gezeigt werden, dass ein perfektes Matching in polynomieller Zeit verifiziert werden kann. Dies ist möglich, indem geprüft wird, ob jedes Element genau einmal in der Tupelmenge auftaucht.

Nun zeigen wir, dass das 4D-Matching-Problem NP-vollständig ist, also dass 3D-Matching $\leq_p$ 4D-Matching gilt. Wir reduzieren dazu das 3D-Matching-Problem auf das 4D-Matching-Problem. Wir zeigen nun eine Konstruktion, welche ein korrektes Matching in 3D auf ein korrektes Matching in 4D, sowie ein falsches Matching in 3D auf ein falsches Matching in 4D abbildet. Dazu verbinden wir die Mengen $C$ und $D$ mithilfe einer bijektiven Abbildung. Man könnte zum Beispiel die Elemente der beiden Mengen jeweils nach einem bestimmten Kriterium sortieren und dann die jeweils $i$-ten Elemente in der Abbildung verbinden (für $i \in \{1,\dots,n\}$). Gegeben sei nun ein perfektes 3D-Matching für die Mengen $A$, $B$ und $C$. Alle Elemente der genannten Mengen sind also in genau einem Tupel enthalten. Nach unserer Konstruktion eines 4D-Matchings ist wegen der Bijektivität in dem Matching genau einmal jedes Element der Menge $D$ vorhanden. Ein falsches Matching in 3D trifft entweder nicht alle Elemente der Mengen $A$ oder $B$, oder nicht alle Elemente aus der Menge $C$ und damit auch der Menge $D$. Somit ist gezeigt, dass sich das 3D-Matching-Problem auf das 4D-Matching-Problem reduzieren lässt, was bedeutet, dass 4D-Matching NP-vollständig ist.


\subsubsection*{b)}
$k$-CLIQUE ist für jedes $2 < k < |V|$ NP-Vollständig. 
\\\\
Beweis: $k$-Clique kann wie folgt auf Independent-Set reduziert werden.
Zu jedem Graphen $G=(V,E)$ wird der komplementäre Graph $G^*=(V^*,E^*)$ erzeugt, wobei $V^*=V$ und $e \in E^*$ genau dann, wenn $e \notin E$ gilt. Anschließend wird auf dem Graphen $G^*$ Indipendent Set ausgeführt. Wenn nun eine unabhängige Menge der größe $k$ in $G^*$ gefunden wird, muss es auch einen vollständigen Graphen der größe $k$ in $G$ geben.
Für $k=|V|$ wird geprüft, ob der Graph $G$ vollständig ist. Dies ist jedoch in $P$ möglich. Da nicht bekannt ist ob $P=NP$ gilt kann keine Aussage darüber getroffen werden, ob $k$-Clique auch für $k=|V|$ $NP$ vollständig ist. Ähnliches gilt für $k=2$. In diesem Fall wird geprüft, ob der Graph $G$ eine Kante enthält.

\subsection*{Aufgabe 2}
\subsubsection*{a)}
Um eine Menge $H$ in polynomieller Zeit als Hitting Set zu verifizieren wird $H$ als Zertifikat verwendet.
Zunächst wird geprüft ob $|H|\le k$ gilt. Als nächstes wird geprüft, ob $|H|\subseteq A$ gilt und zum Schluss wird geprüft, ob $H \cap B_i \neq \emptyset$ für jedes $i \in {1,...,m}$ gilt.
Diese drei Prüfungen liegen in polynomieller Laufzeit. Somit liegt Hitting Set in NP.

\subsubsection*{b)}
Vertex Cover
\end{document}