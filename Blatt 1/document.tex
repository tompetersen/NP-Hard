\documentclass[12pt,a4paper]{article}
\usepackage[latin1]{inputenc}
\usepackage{amsmath}
\usepackage{amsfonts}
\usepackage{amssymb}
\usepackage{graphicx}
\usepackage{color}
\usepackage{listings} 
 \definecolor{lightgrey}{rgb}{0.90,0.90,0.90}
\lstset{language=Java, backgroundcolor=\color{lightgrey},  numbersep=5pt, tabsize=3}
 \usepackage{lineno}
\title{L�sungsstrategien f�r NP-schwere Probleme}
\begin{document}
\section*{NP-schwere Probleme Aufgabenblatt 1}
\subsubsection*{Thomas Maier	(6319878)}
\subsubsection*{Autor2	(Matrikelnr)}
\subsubsection*{Autor3	(Matrikelnr)}
\subsubsection*{Autor4	(Matrikelnr)}

\subsection*{Aufgabe 1}

\subsubsection*{a)}
Zu erst wird gezeigt, dass 4D-Matching in NP liegt. Dazu muss gezeigt werden, dass ein perfektes Matching in polynomieller Zeit verifiziert werden kann. Dies ist M�glich, indem gepr�ft wird ob jedes Element genau einmal in der Tupelmenge auftaucht.
\\\\
Nun bleibt zu zeigen, dass 4D-Matching auch NP-Vollst�ndig ist. Dazu wird das Problem auf das 3D-Matching wie folgt reduziert. Das Ziel ist es ein perfektes Matching der folgenden vier Mengen $A, B, C, D$ zu finden. Daf�r wird zun�chst versucht ein perfektes 3D-Matching $T$ zwischen den Mengen $A,B,C$ zu finden. Ist dies nicht m�glich, kann auch kein perfektes 4D-Matching gefunden werden, da ein solches durch das Entfernen des letzten Elementes eines jeden Tupels auf ein perfektes 3D-Matching zur�ckgef�hrt werden k�nnte.  Nachdem also ein solches perfektes 3D Matching gefunden wurde, wird jedem Tupel aus $T$ ein Element aus $D$ zugeordnet. Wenn jedem Element aus D genau ein Tupel zugeordnet werden kann, erh�lt man dadurch ein perfektes 4D-Matching.

\subsubsection*{b)}
$k$-CLIQUE ist f�r jedes $2 < k < |V|$ NP-Vollst�ndig. 
\\\\
Beweis: $k$-Clique kann wie folgt auf Independet-Set reduziert werden.
Zu jedem Graphen $G=(V,E)$ wird der komplement�re Graph $G^*=(V^*,E^*)$ erzeugt, wobei $V^*=V$ und $e \in E^*$ genau dann, wenn $e \notin E$ gilt. Anschlie�end wird auf dem Graphen $G^*$ Indipendent Set ausgef�hrt. Wenn nun eine unabh�ngige Menge der gr��e $k$ in $G^*$ gefunden wird, muss es auch einen vollst�ndigen Graphen der gr��e $k$ in $G$ geben.
F�r $k=|V|$ wird gepr�ft, ob der Graph $G$ vollst�ndig ist. Dies ist jedoch in $P$ m�glich. Da nicht bekannt ist ob $P=NP$ gilt kann keine Aussage dar�ber getroffen werden, ob $k$-Clique auch f�r $k=|V|$ $NP$ vollst�ndig ist. �hnliches gilt f�r $k=2$. In diesem Fall wird gepr�ft, ob der Graph $G$ eine Kante enth�lt.

\subsection*{Aufgabe 2}
\subsubsection*{a)}
Um eine Menge $H$ in polynomieller Zeit als Hitting Set zu verifizieren wird $H$ als Zertifikat verwendet.
Zun�chst wird gepr�ft ob $|H|\le k$ gilt. Als n�chstes wird gepr�ft, ob $|H|\subseteq A$ gilt und zum Schluss wird gepr�ft, ob $H \cap B_i \neq \emptyset$ f�r jedes $i \in {1,...,m}$ gilt.
Diese drei Pr�fungen liegen in polynomieller Laufzeit. Somit liegt Hitting Set in NP.

\subsubsection*{b)}
Vertex Cover
\end{document}