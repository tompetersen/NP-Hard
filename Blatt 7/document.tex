\documentclass[12pt,a4paper]{article}
\usepackage[utf8]{inputenc}
\usepackage{ngerman}
\usepackage{amsmath}
\usepackage{amsfonts}
\usepackage{amssymb}
\usepackage{graphicx}
\usepackage{color}
\usepackage{enumerate}
\usepackage{lineno}
\usepackage{listings} 
\definecolor{lightgrey}{rgb}{0.90,0.90,0.90}
\lstset{language=Java, backgroundcolor=\color{lightgrey},  numbersep=5pt, tabsize=3}

\setlength{\parindent}{0em}
\setlength{\parskip}{0.5em}

\title{Lösungsstrategien für NP-schwere Probleme\\Blatt 7}
\author{
		Jakob Rieck\\
		\small{6423721}
	\and
		Konstantin Kobs\\
		\small{6414943}
	\and
		Thomas Maier\\
		\small{6319878}
	\and
		Tom Petersen\\
		\small{6359640}
}
\date{Abgabe zum 06.06.16}


\begin{document}

\maketitle

\section*{Aufgabe 1}


\section*{Aufgabe 2}

 \begin{enumerate}[a)]
 	\item Dynamic Programming 1: Waagerecht ist das Gewicht aufgetragen; vertikal die Items; Einträge in der Tabelle sind die (summierten) Werte der Items.
 		
 		\begin{table}[hbt]
  		\begin{tabular}{l|cccccccccc}
  	 	 	4 & 0 & 2 & 2 & 3 & 3 & 3 & 5 & 7 & 7 & \underline{8} \\
	    	3 & 0 & 2 & 2 & 3 & 3 & 3 & 5 & 7 & 7 & \underline{8} \\
	    	2 & 0 & 0 & 1 & 1 & 1 & 1 & 5 & 5 & \underline{6} & \underline{6} \\
	    	1 & 0 & 0 & 0 & 0 & 0 & 0 & \underline{5} & 5 & \underline{5} & 5 \\
	    	0 & 0 & 0 & 0 & 0 & 0 & 0 & \underline{0} & 0 & 0 & 0 \\
	    	\hline
	    	  & 0 & 1 & 2 & 3 & 4 & 5 & 6 & 7 & 8 & 9 \\
  		\end{tabular}
		\end{table}
		
		Die Tabelle zeigt, dass wir einen maximalen Wert von 8 erreichen können. Hierzu müssen wir die Items 1, 2 und 3 in den Rucksack packen.

 	
 	\item Dynamic Programming 2: Waagerecht ist der maximale Gesamtwert (11) aufgetragen; vertikal die Items; Einträge in der Tabelle sind die (summierten) Gewichte.
 		
 		\begin{table}[hbt]
  		\begin{tabular}{l|cccccccccccc}
  	 	 	4 & 0 & 1 & 1 & 3 & 6 & 6 & 7 & 7 & \underline{9} & 12 & 12 & 14\\
	    	3 & 0 & 1 & 1 & 3 & 6 & 6 & 7 & 7 & \underline{9} & $\infty$ & $\infty$ & $\infty$\\
	    	2 & 0 & 2 & 6 & 6 & 6 & 6 & \underline{8} & $\infty$ & \underline{$\infty$} & $\infty$ & $\infty$ & $\infty$\\
	    	1 & 0 & 6 & 6 & 6 & 6 & \underline{6} & \underline{$\infty$} & $\infty$ & $\infty$ & $\infty$ & $\infty$ & $\infty$\\
	    	0 & 0 & $\infty$ & $\infty$ & $\infty$ & $\infty$ & \underline{$\infty$} & $\infty$ & $\infty$ & $\infty$ & $\infty$ & $\infty$ & $\infty$\\
	    	\hline
	    	  & 0 & 1 & 2 & 3 & 4 & 5 & 6 & 7 & 8 & 9 & 10 & 11\\
  		\end{tabular}
		\end{table}
 	 
 		Die Tabelle zeigt, dass wir ein Gewicht von maximal 9 erreichen, wobei wir einen Wert von 8 erreichen. Hierzu müssen, wie schon in \textit{a)}, Items 1, 2 und 3 hinzugefügt werden. Dies ist das gleiche Ergebnis wie in \textit{a)}, denn schließlich handelt es sich hier nur um zwei verschiedene Berechnungsweisen des gleichen Problems.
 	
\end{enumerate}


\end{document}