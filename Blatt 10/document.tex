\documentclass[12pt,a4paper]{article}
\usepackage[utf8]{inputenc}
\usepackage{ngerman}
\usepackage{amsmath}
\usepackage{amsfonts}
\usepackage{amssymb}
\usepackage{graphicx}
\usepackage{color}
\usepackage{enumerate}
\usepackage{lineno}
\usepackage{listings} 
\definecolor{lightgrey}{rgb}{0.90,0.90,0.90}
\lstset{language=Java, backgroundcolor=\color{lightgrey},  numbersep=5pt, tabsize=3}

\setlength{\parindent}{0em}
\setlength{\parskip}{0.5em}

\title{Lösungsstrategien für NP-schwere Probleme\\Blatt 10}
\author{
		Jakob Rieck\\
		\small{6423721}
	\and
		Konstantin Kobs\\
		\small{6414943}
	\and
		Thomas Maier\\
		\small{6319878}
	\and
		Tom Petersen\\
		\small{6359640}
}
\date{Abgabe zum 04.07.16}


\begin{document}

\maketitle

\section*{Aufgabe 1}

Bei der Zufallsgröße handelt es sich um zwei voneinander unabhängige Zufallsverteilungen. Zum einen werden in die Zufallsgröße $X$ alle Stimmen der Leute gezählt, die für $D$ stimmen wollten und es auch taten. Dies geschieht mit Wahrscheinlichkeit $\frac{99}{100}$ bei allen 80000 $D$-Fans. Zum anderen beinhaltet $X$ aber auch die Stimmen der Leute, die $R$ wählen wollten, allerdings aufgrund der schlecht designten Wahlzettel $D$ wählten. Dies geschieht mit Wahrscheinlichkeit $\frac{1}{100}$ für die gegebenen 20000 $R$-Fans.

Beide Wahrscheinlichkeitsverteilungen sind Binomialverteilungen, die zum einen ein $n=80000$ und zum anderen ein $n=20000$ sowie die oben genannten Wahrscheinlichkeiten $p=\frac{99}{100}$ sowie $p=\frac{1}{100}$ besitzen. Die Zufallsgröße $X$ wird nun beschrieben als

$$X = Binom(80000, \frac{99}{100}) + Binom(20000, \frac{1}{100})$$

Somit ist der Erwartungswert von $X$

$$E[X] = E[Binom(80000, \frac{99}{100}) + Binom(20000, \frac{1}{100})]$$

Wegen der Linearität des Erwartungswertes gilt dann

$$E[X] = E[Binom(80000, \frac{99}{100})] + E[Binom(20000, \frac{1}{100})]$$

Für Binomialverteilungen ist der Erwartungswert leicht zu berechnen ($n \cdot p$), weshalb der Erwartungswert folgendes ist:

$$E[X] = 80000 \cdot \frac{99}{100} + 20000 \cdot \frac{1}{100} = 79400$$


\section*{Aufgabe 2}

\begin{enumerate}[a)]
	\item Wenn ein Prozess einen Wert von 1 wählt, und gleichzeitig alle verbundenen Prozesse den Wert 0 haben, so kann der Prozess in die Menge $S$ aufgenommen werden. In der Menge $S$ befinden sich am Ende also nur Prozesse, die nicht im Konflikt zu ihren möglichen Konflikt-Partnern stehen, da alle diese nicht in die Menge aufgenommen wurden. $\square$\\
		Die Wahrscheinlichkeit, dass ein Prozess $P$ den Wert 1 wählt, liegt bei $\frac{1}{2}$. Dieselbe Wahrscheinlichkeit hat auch jeder andere Prozess, der mit $P$ verbunden ist. Die Wahrscheinlichkeit, dass nun $P$ den Wert 1 hat \textbf{und} die $d$ mit ihm verbundenen Prozesse den Wert 0 liegt bei $\frac{1}{2} \cdot \frac{1}{2}^d$, da die beiden Ereignisse unabhängig voneinander und identisch verteilt sind (iid). Da dies nun die Wahrscheinlichkeit für ein Element ist, und es immer nur zwei Zustände gibt (in Set $S$ oder nicht), handelt es sich hier um eine Binomialverteilung, deren Erwartungswert wir dann einfach mit
		$$\frac{1}{2} \cdot \frac{1}{2}^d \cdot n = \frac{1}{2}^{d+1} \cdot n$$
		beschreiben können.

	\item Die Wahrscheinlichkeitsberechnung ist hier ähnlich, nur etwas allgemeiner. Die Wahrscheinlichkeit für den Wert 1 ist $p$ und dass die in Konflikt stehenden Prozesse jeweils 0 sind, hat die Wahrscheinlichkeit $1-p$. Für die Gesamtwahrscheinlichkeit ergibt sich dann $p \cdot (1-p)^d$, sodass sich auch hier eine Binomialverteilung ergibt mit dem Erwartungswert
		$$p \cdot (1-p)^d \cdot n$$
		
		TODO: Für welches $p$ ist das maximal? Ableiten ist schwierig.
		
\end{enumerate}





\end{document}