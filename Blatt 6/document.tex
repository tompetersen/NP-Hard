\documentclass[12pt,a4paper]{article}
\usepackage[utf8]{inputenc}
\usepackage{ngerman}
\usepackage{amsmath}
\usepackage{amsfonts}
\usepackage{amssymb}
\usepackage{graphicx}
\usepackage{color}
\usepackage{enumerate}
\usepackage{lineno}
\usepackage{listings} 
\definecolor{lightgrey}{rgb}{0.90,0.90,0.90}
\lstset{language=Java, backgroundcolor=\color{lightgrey},  numbersep=5pt, tabsize=3}

\setlength{\parindent}{0em}
\setlength{\parskip}{0.5em}

\title{Lösungsstrategien für NP-schwere Probleme\\Blatt 6}
\author{
		Jakob Rieck\\
		\small{6423721}
	\and
		Konstantin Kobs\\
		\small{6414943}
	\and
		Thomas Maier\\
		\small{6319878}
	\and
		Tom Petersen\\
		\small{6359640}
}
\date{Abgabe zum 23.05.16}


\begin{document}

\maketitle

\section*{Aufgabe 1}

 \begin{enumerate}[a)]

 	\item Ein Beispiel, an dem leicht zu sehen ist, dass es sich bei dem beschriebenen Algorithmus um keinen \(2\)-Approximationsalgorithmus handelt, ist das folgende:\\	
 	 Es seien \(A = \{a_1, a_2\}\) mit \(a_1 = 1, a_2 = 3\) und \(B = 3\). Offensichtlich ist die optimale Lösung \(S = \{a_2\}\) mit der totalen Summe \(3\). Der Algorithmus in der vorgegebenen Form würde jedoch im ersten Schritt \(a_1\) zu \(S\) hinzufügen und dieses - da \(a_2\) nicht mehr hinzugenommen werden darf - als Lösung zurückliefern. 
 	 
 	 Für einen \(2\)-Approximationsalgorithmus müsste jedoch \(\frac{L^*}{L} \le 2\) gelten (da es sich um ein Maximierungsproblem handelt). Hier ergibt sich jedoch \(\frac{L^*}{L} = \frac{3}{1} > 2\). Damit handelt es sich bei dem Algorithmus um keinen \(2\)-Approximationsalgorithmus.
 	
 	\item Der Algorithmus sortiert im ersten Schritt die Menge \(A\) der Größe nach absteigend (Laufzeit in der Praxis \(\mathcal{O}(n \log n)\)). Anschließend wird der Algorithmus aus Teilaufgabe a) auf die sortierte Folge angewendet (Laufzeit \(\mathcal{O}(n)\)).
 	
 	Im Folgenden soll nun gezeigt werden, dass es sich bei dem Verfahren um einen \(2\)-Approximationsalgorithmus handelt.\\
 	Anfangselemente der sortierten Folge für die \(a_i > B\) gilt, können vernachlässigt werden, da sie in keinem Fall in der gesuchten Menge \(S\) auftreten können. 
 	
 	Betrachtet wird nun das erste zu \(S\) hinzugenommene Element \(a_b\).\\
 	Falls \(a_b \ge \frac{B}{2}\) gilt, so handelt es sich auf jeden Fall um einen \(2\)-Appro\-xi\-ma\-tions\-algorithmus, da \(\frac{L^*}{L} = \frac{B}{L} \le \frac{B}{\frac{B}{2}} = 2\) (\(B\) ist für das betrachtete Problem der Wert einer optimalen Lösung).\\
 	Anderenfalls gilt \(a_b < \frac{B}{2}\). Dann kann in der Folge kein \(a_j\) mit \(\frac{B}{2} < a_j \le B\) existieren, da es ansonsten bereits vorher in \(S\) aufgenommen worden wäre. Anschließend können nun weitere Elemente \(a_k\) in \(S\) aufgenommen werden, wegen \(a_k \le a_b < \frac{B}{2}\). Wenn \(T\) irgendwann den Wert \(\frac{B}{2}\) überschreitet, so liegt wegen \(\frac{L^*}{L} = \frac{B}{L} \le \frac{B}{\frac{B}{2}} = 2\) ein \(2\)-Appro\-xi\-ma\-tions\-algorithmus vor. Findet diese Überschreitung nicht statt, so können alle Elemente aus \(A\) in \(S\) aufgenommen werden und die gefundene Lösung ist sogar eine optimale. 
 	
 	Insgesamt handelt es sich bei dem Verfahren also um einen \(2\)-Appro\-xi\-ma\-tions\-algorithmus.
 	
 	
\end{enumerate}

\section*{Aufgabe 2}

Berechnen wir zunächst für das Beispiel-Problem eine optimale Lösung. Dazu bilden wir den Durchschnitt der Gesamtlast für jede Maschine. Wir können hierbei den Satz von Gauss ("`kleiner Gauss"') anwenden und erhalten:

\begin{align*}
	L_{\text{gesamt}} &= 2 \cdot \sum_{i=m}^{2m-1}(i) + m\\
	&= m \cdot (3m - 1) + m\\
	&= 3m^2\\
	L^* &= \frac{3m^2}{m} = 3m
\end{align*}

In der (möglicherweise gar nicht erreichbaren) optimalen Lösung hat also jede Maschine eine Last von $3m$.

Nun betrachten wir Greedy-Balance mit LIF-Regel. Da die Jobs zunächst nach ihren Lasten sortiert werden, können wir uns zunächst nur $2m$ Jobs anschauen, wobei wir den kleinsten Job mit Last $m$ noch nicht betrachten. Während des Algorithmus werden nacheinander die Jobs zugeordnet. Da die Summe zweier Job-Lasten immer größer ist als eine einzelne Job-Last, werden zunächst alle Maschinen mit genau einem Job belegt. Weil die kleineren Jobs später hinzugefügt wurden, gibt es dort noch weniger Auslastungen, sodass die nächsten Jobs in der umgedrehten Reihenfolge auf die Maschinen verteilt werden. Somit besitzt jede Maschine nach den ersten $2m$ Jobs eine Last von $3m-1$. Der letzte Job mit Last $m$ muss nun einer Maschine zugeteilt werden. Somit hat diese Maschine eine Last von $(3m-1)+m = 4m-1$, was bedeutet, dass diese Maschine auch die größte Last aller Maschinen hat. $L_A$, die gefundene Lösung des Algorithmus, ist also $4m-1$. Der Quotient $\frac{L_A}{L^*}$ ist also

$$\frac{L_A}{L^*} = \frac{4m-1}{3m} = \frac{4}{3} - \frac{1}{3m}$$

was mit wachsendem $m$ gegen $\frac{4}{3} = \rho$ strebt. Damit ist gezeigt, dass $\frac{4}{3}$ die bestmögliche Gütegarantie für dieses Beispiel ist.


\end{document}